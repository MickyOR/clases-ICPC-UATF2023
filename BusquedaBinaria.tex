\documentclass[10pt]{beamer}
\usepackage[utf8]{inputenc}
\usepackage[T1]{fontenc}
\usetheme{metropolis}
\usepackage{booktabs}
\usepackage[scale=2]{ccicons}
\usepackage{pgfplots}
\usepgfplotslibrary{dateplot}
\usepackage{xspace}
\usepackage{pbox}
\usepackage[normalem]{ulem}

% graphics path
\graphicspath{{./images/}}

% a few macros
\newcommand{\bi}{\begin{itemize}}
\newcommand{\ei}{\end{itemize}}
\newcommand{\ig}{\includegraphics}
\newcommand{\x}{\color{red}{$\times$}}
\definecolor{hilight}{RGB}{235,129,27}
\definecolor{vhilight}{RGB}{235,129,27}

% title info
\title{Búsqueda binaria}
\author{Miguel Ortiz}
\institute{Octubre 2023}
\date{\textbf{Programación competitiva para ICPC}}

% Tikz
\usepackage{tikz}
\usetikzlibrary{arrows,shapes}

% Minted
\usepackage{minted}
\usemintedstyle{manni}
\newminted{cpp}{fontsize=\footnotesize}

% Graph styles
\tikzstyle{vertex}=[circle,fill=black!50,minimum size=15pt,inner sep=0pt, font=\small]
\tikzstyle{selected vertex} = [vertex, fill=red!24]
\tikzstyle{edge} = [draw,thick,-]
\tikzstyle{dedge} = [draw,thick,->]
\tikzstyle{weight} = [font=\scriptsize,pos=0.5]
\tikzstyle{selected edge} = [draw,line width=2pt,-,red!50]
\tikzstyle{ignored edge} = [draw,line width=5pt,-,black!20]


\begin{document}
\maketitle

\begin{frame}{Problema de ejemplo}
  Dado un arreglo \textbf{ordenado} de enteros, indicar si contiene un elemento \textbf{$val$} o no.
  \vspace{20pt}

  \onslide<2->{Algoritmo:}
  \bi
    \item<2-> Revisamos el elemento del medio
    \bi
      \item<2-> Si es $val$, terminamos
      \item<2-> Si es mayor que $val$, buscamos en la mitad izquierda
      \item<2-> Si es menor que $val$, buscamos en la mitad derecha
    \ei
    \item<3-> Hacemos lo mismo en las mitades más pequeñas hasta encontrar el número o hasta que no queden más elementos
    \item<4-> Primero hay $n$ candidatos, luego $\frac{n}{2}$, luego $\frac{n}{4}$, etc.
    \item<4-> $O(\log n)$
  \ei
\end{frame}

\begin{frame}{Problema de ejemplo}
  $val = 22$

  \vspace{20pt}

  \begin{center}
    \begin{overprint}
      \onslide<1>\begin{tabular}{|ccccccccccccc|}
        \;1 & \;4 & \;5 & \;8 & 11 & 19 & \color{vhilight}{21} & 22 & 25 & 29 & 31 & 37 & 42 \\
        \hline
        lo &  &  &  &  &  & \color{vhilight}{m} &  &  &  &  &  & hi \\
      \end{tabular}
      \onslide<2>\begin{tabular}{ccccccc|cccccc|}
        \;1 & \;4 & \;5 & \;8 & 11 & 19 & 21 & 22 & 25 & \color{vhilight}{29} & \color{vhilight}{31} & 37 & 42 \\
        \hline
        \x & \x & \x & \x & \x & \x & \x & lo & & \color{vhilight}{m} & \color{vhilight}{m} & & hi \\
      \end{tabular}
      \onslide<3>\begin{tabular}{ccccccc|cccccc|}
        \;1 & \;4 & \;5 & \;8 & 11 & 19 & 21 & 22 & 25 & \color{vhilight}{29} & 31 & 37 & 42 \\
        \hline
        \x & \x & \x & \x & \x & \x & \x & lo & & \color{vhilight}{m} & & & hi \\
      \end{tabular}
      \onslide<4>\begin{tabular}{ccccccc|cc|cccc}
        \;1 & \;4 & \;5 & \;8 & 11 & 19 & 21 & \color{vhilight}{22} & 25 & 29 & 31 & 37 & 42 \\
        \hline
        \x & \x & \x & \x & \x & \x & \x & lo & hi & \x & \x & \x & \x \\
         &  &  &  &  &  &  & \color{vhilight}{m} &  &  &  &  &  \\
      \end{tabular}
      \onslide<5>\begin{tabular}{ccccccc|cc|cccc}
        \;1 & \;4 & \;5 & \;8 & 11 & 19 & 21 & \textbf{\color{teal}{22}} & 25 & 29 & 31 & 37 & 42 \\
        \hline
        \x & \x & \x & \x & \x & \x & \x & lo & hi & \x & \x & \x & \x \\
         &  &  &  &  &  &  & \color{vhilight}{m} &  &  &  &  &  \\
      \end{tabular}
    \end{overprint}
  \end{center}
\end{frame}

\begin{frame}{Problema de ejemplo}
  $val = 23$

  \vspace{20pt}

  \begin{center}
    \begin{overprint}
      \onslide<1>\begin{tabular}{|ccccccccccccc|}
        \;1 & \;4 & \;5 & \;8 & 11 & 19 & \color{vhilight}{21} & 22 & 25 & 29 & 31 & 37 & 42 \\
        \hline
        lo &  &  &  &  &  & \color{vhilight}{m} &  &  &  &  &  & hi \\
      \end{tabular}
      \onslide<2>\begin{tabular}{ccccccc|cccccc|}
        \;1 & \;4 & \;5 & \;8 & 11 & 19 & 21 & 22 & 25 & \color{vhilight}{29} & 31 & 37 & 42 \\
        \hline
        \x & \x & \x & \x & \x & \x & \x & lo & & \color{vhilight}{m} & & & hi \\
      \end{tabular}
      \onslide<3>\begin{tabular}{ccccccc|cc|cccc}
        \;1 & \;4 & \;5 & \;8 & 11 & 19 & 21 & \color{vhilight}{22} & 25 & 29 & 31 & 37 & 42 \\
        \hline
        \x & \x & \x & \x & \x & \x & \x & lo & hi & \x & \x & \x & \x \\
         &  &  &  &  &  &  & \color{vhilight}{m} &  &  &  &  &  \\
      \end{tabular}
      \onslide<4>\begin{tabular}{cccccccc|c|cccc}
        \;1 & \;4 & \;5 & \;8 & 11 & 19 & 21 & 22 & \color{vhilight}{25} & 29 & 31 & 37 & 42 \\
        \hline
        \x & \x & \x & \x & \x & \x & \x & \x & lo & \x & \x & \x & \x \\
         &  &  &  &  &  &  &  & hi &  &  &  &  \\
         &  &  &  &  &  &  &  & \color{vhilight}{m} &  &  &  &  \\
      \end{tabular}
      \onslide<5>\begin{tabular}{ccccccccccccc}
        \;1 & \;4 & \;5 & \;8 & 11 & 19 & 21 & 22 & 25 & 29 & 31 & 37 & 42 \\
        \hline
        \x & \x & \x & \x & \x & \x & \x & \x & \x & \x & \x & \x & \x \\
         &  &  &  &  &  &  & \color{red}{hi} & \color{red}{lo} & &  &  &  \\
      \end{tabular}
    \end{overprint}
  \end{center}
\end{frame}

\begin{frame}[fragile]{Problema de ejemplo}
  \begin{minted}{cpp}
    ... // leer n, val, arreglo a
    int lo = 0, hi = n-1;
    int res = -1;
    while (lo <= hi) {
      int mid = (lo + hi) / 2;
      if (a[mid] == val) {
        res = mid;
        break;
      }
      else if (val < a[mid]) {
        hi = mid - 1;
      }
      else { // val > a[mid]
        lo = mid + 1;
      }
    }
    // val esta en la posicion res, -1 si no esta
  \end{minted}
\end{frame}

\begin{frame}{Forma general}
  \bi
    \item Trabajar sobre un rango de enteros
    \item $
      f(x) = \begin{cases}
        1 & \text{si } x \text{ cumple la propiedad} \\
        0 & \text{si } x \text{ \textbf{no} cumple la propiedad}
      \end{cases}
    $
    \item $f(x)$ es monótona $\star$
  \ei

  \onslide<2->\begin{center}
    \begin{tabular}{cc}
      $val = 22$ & $
      f(x) = \begin{cases}
        1 & \text{si } x \geq val \\
        0 & \text{si } x < val
      \end{cases}
    $
    \end{tabular}
  \end{center}
  
  \onslide<2->\begin{center}
    \begin{tabular}{ccccccccccccccccccc}
      \;1 & \;4 & \;5 & \;8 & 11 & 19 & 21 & 22 & 25 & 29 & 31 & 37 & 42 \\
      \hline
      0 & 0 & 0 & 0 & 0 & 0 & 0 & 1 & 1 & 1 & 1 & 1 & 1 \\
    \end{tabular}
  \end{center}

  \bi
    \onslide<3->\item Podemos buscar el primer 1 o el último 0
  \ei
\end{frame}

\begin{frame}[fragile]{Forma general}
  \begin{minted}{cpp}
    int lo = 0, hi = n-1;
    int res = -1;
    while (lo <= hi) {
      int mid = (lo + hi) / 2;
      if (f(mid) == 1) {
        res = mid; // mejor respuesta hasta ahora
        hi = mid - 1;
      }
      else { // f(mid) == 0
        lo = mid + 1;
      }
    }
    // Primer 1 esta en la posicion res
  \end{minted}
\end{frame}

\begin{frame}{Forma general}
  $val = 25$

  \vspace{20pt}

  \begin{center}
    \begin{overprint}
      \onslide<1>\begin{tabular}{|ccccccccccccc|}
        \;1 & \;4 & \;5 & \;8 & 11 & 19 & \color{vhilight}{21} & 22 & 25 & 29 & 31 & 37 & 42 \\
        \hline
        lo &  &  &  &  &  & \color{vhilight}{m} &  &  &  &  &  & hi \\
      \end{tabular}
      \onslide<2>\begin{tabular}{ccccccc|cccccc|}
        \;1 & \;4 & \;5 & \;8 & 11 & 19 & 21 & 22 & 25 & \color{vhilight}{29} & 31 & 37 & 42 \\
        \hline
        \x & \x & \x & \x & \x & \x & \x & lo & & \color{vhilight}{m} & & & hi \\
         & & &  &  & &  & & & \color{vhilight}{res} & & &  \\
      \end{tabular}
      \onslide<3>\begin{tabular}{ccccccc|cc|cccc}
        \;1 & \;4 & \;5 & \;8 & 11 & 19 & 21 & \color{vhilight}{22} & 25 & 29 & 31 & 37 & 42 \\
        \hline
        \x & \x & \x & \x & \x & \x & \x & lo & hi & \x & \x & \x & \x \\
         &  &  &  &  &  &  & \color{vhilight}{m} &  &  &  &  &  \\
         & & &  &  & &  & & & \color{vhilight}{res} & & &  \\
      \end{tabular}
      \onslide<4>\begin{tabular}{cccccccc|c|cccc}
        \;1 & \;4 & \;5 & \;8 & 11 & 19 & 21 & 22 & \color{vhilight}{25} & 29 & 31 & 37 & 42 \\
        \hline
        \x & \x & \x & \x & \x & \x & \x & \x & lo & \x & \x & \x & \x \\
         &  &  &  &  &  &  &  & hi &  &  &  &  \\
         &  &  &  &  &  &  &  & \color{vhilight}{m} &  &  &  &  \\
         & & &  &  & &  & & \color{vhilight}{res} & \sout{res} & & &  \\
      \end{tabular}
      \onslide<5>\begin{tabular}{ccccccccccccc}
        \;1 & \;4 & \;5 & \;8 & 11 & 19 & 21 & 22 & \color{teal}{25} & 29 & 31 & 37 & 42 \\
        \hline
        \x & \x & \x & \x & \x & \x & \x & \x & \x & \x & \x & \x & \x \\
         &  &  &  &  &  &  & \color{red}{hi} & \color{red}{lo} &  &  &  &  \\
         & & &  &  & &  & & \color{teal}{res} & \sout{res} & & &  \\
      \end{tabular}
    \end{overprint}
  \end{center}
\end{frame}

\begin{frame}{Problema de ejemplo}
  Tenemos un arreglo $a$ que contiene $n$ enteros.

  Se deben responder $q$ consultas de la forma: Dados dos valores $l$ y $r$, 
  ¿cuantos elementos del arreglo $a$ están en el rango $[l, r]$?

  \bi
    \item<2-> No hay ninguna restriccion sobre el orden de los elementos. Podemos ordenar el arreglo para 
    encontrar una función monótona
    \item<3-> Para encontrar limite izquierdo, buscamos el primer elemento mayor o igual que $l$
    \item<3-> Función monótona $0, 0, ..., 0, 0, 0, 1, 1, 1, ...$
    \item<4-> Para encontrar el limite derecho, buscamos el ultimo elemento menor o igual que $r$
    \item<4-> Función monótona $1, 1, ..., 1, 1, 1, 0, 0, 0, ...$
  \ei
\end{frame}

\begin{frame}{Problema de ejemplo}
  Tenemos $n$ piezas rectangulares de tamaño $a \times b$. 
  ¿Cuál es el tamaño más pequeño que puede tener el lado de un cuadrado que contenga a todas las piezas?

  Las piezas no se pueden rotar ni superponer. El tamaño del cuadrado debe ser entero.

  \bi
    \item<2-> Si las $n$ piezas entran en un cuadrado de lado $x$, entrarán en uno de lado $x+1$
    \item<2-> Si las $n$ piezas no entran en un cuadrado de lado $x$, no entrarán en uno de lado $x-1$
    \item<3-> Función monótona $0, 0, ..., 0, 0, 0, 1, 1, 1, ...$
    \item<4-> A esta forma de aplicar búsqueda binaria se le llama \textbf{binary search the answer}.
  \ei
\end{frame}

\begin{frame}{Problema de ejemplo}
  Un cuadrado de $x \cdot x$ puede:
  \bi
    \item acomodar $\lfloor \frac{x}{a} \rfloor$ piezas de un lado
    \item acomodar $\lfloor \frac{x}{b} \rfloor$ piezas del otro lado
  \ei
  \begin{center}
    \ig[width=0.6\textwidth]{ejemploCuadrado.png}
  \end{center}
  \begin{center}
    $
      f(x) = \begin{cases}
        1 & \text{si } \lfloor \frac{x}{a} \rfloor \cdot \lfloor \frac{x}{b} \rfloor \geq n \\
        0 & \text{si } \lfloor \frac{x}{a} \rfloor \cdot \lfloor \frac{x}{b} \rfloor < n
      \end{cases}
    $
  \end{center}
\end{frame}

\begin{frame}{Problema de ejemplo}
  Tenemos $n$ cuerdas, cada una de largo $l_i$. Necesitamos $k$ cuerdas del mismo tamaño.

  ¿Cual es el largo mas grande que pueden tener las $k$ cuerdas si podemos cortar las cuerdas 
  originales de la forma que queramos?

  La respuesta se considera correcta si el error relativo o absoluto no es mayor a \textbf{$10^{-6}$}

  \bi
    \item<2-> Imaginemos que $x$ es la respuesta optima
    \item<3-> Podremos obtener $k$ o mas cuerdas si las cortamos de largo \textbf{menor} que $x$
    \item<3-> Es imposible obtener $k$ o mas cuerdas si las cortamos de largo \textbf{mayor} que $x$
    \item<4-> La funcion es monótona $..., 1, 1, 0, 0, ...$ sobre un rango \textbf{continuo}
  \ei
\end{frame}

\begin{frame}{Problema de ejemplo}
  \begin{center}
      \begin{tabular}{|l|l|}
          \hline
          {\footnotesize Input de ejemplo} & {\footnotesize Output de ejemplo} \\
          \hline
          \begin{minipage}{80pt}
\vspace{10pt}
\ttfamily
4 11\\
802\\
743\\
457\\
539\\
          \end{minipage}
&
\begin{minipage}{80pt}
\vspace{10pt}
\ttfamily
200.5\\
\end{minipage}
\\
          \hline
      \end{tabular}
  \end{center}

\bi
  \item Se pueden obtener al menos 11 cuerdas de largo $200.5$
  \item<2-> Tambien se pueden obtener al menos 11 cuerdas de largo $200.4$, $200.45$, $200.499999$, etc.
  \item<3-> No es necesario que la respuesta sea exacta, pero debe ser un valor muy cercano.
\ei

\end{frame}

\begin{frame}[fragile]{Problema de ejemplo}
  \begin{minted}{cpp}
    ... // leer n, k, arreglo l
    double lo = 0, hi = cuerdaMasLarga;
    double res = -1;
    while (lo <= hi) {
      double mid = (lo + hi) / 2;
      int cuerdas = 0;
      for (int i = 0; i < n; i++) {
        cuerdas += floor(l[i] / mid);
      }
      if (cuerdas >= k) {
        res = mid;
        lo = mid + ???;
      }
      else {
        hi = mid - ???;
      }
    }
  \end{minted}
\end{frame}

\begin{frame}[fragile]{Problema de ejemplo}
  \begin{minted}{cpp}
    ... // leer n, k, arreglo l
    double lo = 0, hi = cuerdaMasLarga;
    for (int aux = 0; aux < 50; aux++) {
      double mid = (lo + hi) / 2;
      int cuerdas = 0;
      for (int i = 0; i < n; i++) {
        cuerdas += floor(l[i] / mid);
      }
      if (cuerdas >= k) lo = mid;
      else hi = mid;
    }
  \end{minted}
  \bi
    \item<2-> Cada iteracion reduce el rango de posibilidades a la mitad
    \item<3-> Luego de 50 iteraciones, el rango sera $\frac{maxCuerda}{2^{50}}$
    \item<4-> Probablemente suficientemente preciso
  \ei
\end{frame}

\end{document}
