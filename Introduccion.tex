\documentclass[10pt]{beamer}
\usepackage[utf8]{inputenc}
\usepackage[T1]{fontenc}
\usetheme{metropolis}
\usepackage{booktabs}
\usepackage[scale=2]{ccicons}
\usepackage{pgfplots}
\usepgfplotslibrary{dateplot}
\usepackage{xspace}
\usepackage{pbox}

% graphics path
\graphicspath{{./images/}}

% a few macros
\newcommand{\bi}{\begin{itemize}}
\newcommand{\ei}{\end{itemize}}
\newcommand{\ig}{\includegraphics}

% title info
\title{Introducción}
\author{Miguel Ortiz - Universidad Autónoma Tomás Frías}
\institute{Octubre 2023}
\date{\textbf{Programación competitiva para ICPC}}

% Tikz
\usepackage{tikz}
\usetikzlibrary{arrows,shapes}

% Minted
\usepackage{minted}
\usemintedstyle{manni}
\newminted{cpp}{fontsize=\footnotesize}

% Graph styles
\tikzstyle{vertex}=[circle,fill=black!50,minimum size=15pt,inner sep=0pt, font=\small]
\tikzstyle{selected vertex} = [vertex, fill=red!24]
\tikzstyle{edge} = [draw,thick,-]
\tikzstyle{dedge} = [draw,thick,->]
\tikzstyle{weight} = [font=\scriptsize,pos=0.5]
\tikzstyle{selected edge} = [draw,line width=2pt,-,red!50]
\tikzstyle{ignored edge} = [draw,line width=5pt,-,black!20]


\begin{document}
\maketitle

\begin{frame}{Mi perfil}
    \bi
        \item Ingeniería Informática en la Universidad Mayor de San Simón
        \item Experiencia en competencias ICPC desde 2016
        \item Clasificado a la final mundial ICPC 2022
        \item Experiencia dando clases de programación competitiva a estudiantes de colegio y universidad
        \vspace{10pt}
        \item \href{https://github.com/MickyOR/}{github.com/MickyOR}
        \item \href{https://t.me/MickyOr}{t.me/MickyOr}
    \ei
\end{frame}

\begin{frame}{¿Qué es la programación competitiva?}
    \begin{columns}[T] % align columns
        \begin{column}{.48\textwidth}
            \bi
                \item Es un deporte mental
                \bi
                    \item Deporte: \textit{Actividad física, 
                    usualmente competitiva y organizada, 
                    que tiene la meta de usar, mantener o mejorar 
                    las capacidades y habilidades físicas.}
                \ei
                \item Se resuelven problemas lógicos
                \item Diseño de algoritmos 
                \item Trabajo en equipo (ICPC)
            \ei
        \end{column}

        \hfill

        \begin{column}{.48\textwidth}
            \ig[width=\textwidth]{ICPC_team.jpg}
        \end{column}
    \end{columns}
\end{frame}

\begin{frame}{Ventajas}
    \bi
        \item Se practica programación
        \item Se mejora el razonamiento algorítmico
        \item Cierra la brecha entre la teoría y la practica
        \item Más oportunidades de trabajo
        \bi
            \item Entrevistas de trabajo
            \item Pasantías en empresas de software
        \ei
    \ei
\end{frame}

\begin{frame}{Meta del curso}
    Dado un problema, queremos:
    \bi
        \item Resolverlo eficientemente, usando algoritmos y estructuras de datos
        \item Convertir la solución a un programa
        \item Hacerlo lo más rápido posible (bajo presión)
        \item Hacerlo correctamente (sin bugs)
    \ei
\end{frame}

\begin{frame}{¿Cómo lo haremos?}
    \bi
        \item Estudiando categorías comunes de problemas
        \item Aprendiendo algoritmos y estructuras de datos
        \item Aprendiendo otros conceptos importantes para resolver problemas
        \item Practicando resolver problemas
        \item Practicando más
        \item ¡Practicando aún más!
    \ei
\end{frame}

\begin{frame}{Requisitos para el curso}
    \bi
        \item Variables
        \item Lectura/escritura de datos por consola
        \item Bucles (\mintinline{cpp}|for, while|)
        \item Condicionales (\mintinline{cpp}|if/else|)
        \item Operadores lógicos
        \item Operadores aritméticos
        \item Funciones
        \bi 
            \item Recursividad (solo para un par de temas, no es 100\% necesario)
        \ei
        \item Arreglos
    \ei
\end{frame}

\begin{frame}{Programa}
    % \scriptsize
    \begin{center}
        \begin{tabular}{cl|l}

            No. de clase & Fecha & Temas/actividad \\
            \hline
            1 & 02.10 & Introducción, análisis de complejidad, \\ 
              &       & estructuras de datos no lineales ya implementadas \\
            2 & 04.10 & Grafos \\
            3 & 06.10 & Búsqueda binaria \\
            \hline
            4 & 09.10 & Matemáticas \\
            5 & 11.10 & Programación dinámica \\
            6 & 13.10 & Estructuras de datos para consultas en rangos \\
            \hline
        \end{tabular}
    \end{center}
    Horario: 20:30 - 22:30
\end{frame}

\begin{frame}{Formato}
    \bi
        \item Clases teóricas: 
        \bi
            \item Resolución de problemas del tema anterior
            \item Explicación de temas
        \ei
        \item Competencias:
        \bi
            \item Duran 46 horas, desde el final de una clase 
            hasta el inicio de la siguiente
            \item Problemas relacionados con el último tema
            \item Pueden participar individualmente o en equipos
            \item Se puede participar de forma individual
        \ei
    \ei
\end{frame}

\section{Introducción a competencias}

\begin{frame}{Problemas}
    \bi
        \item Problemas típicos en competencias de programación
        \item Usualmente consisten de:
            \bi
                \item Descripción del problema
                \item Descripción del input
                \item Descripción del output
                \item Ejemplo de un caso de prueba (input y output)
                \item El tiempo límite en segundos
                \item La memoria límite en MB o kB
            \ei
        \item Te dicen que escribas un programa que resuelva el problema para todos los inputs válidos
        \item El programa no debe exceder los límites de tiempo o memoria
    \ei
\end{frame}

\begin{frame}{Ejemplo de un problema}
    \begin{block}{Descripcion del problema}
    Escriba un programa que multiplique pares de enteros.
    \end{block}

    \vspace{10pt}

    \begin{block}{Input}
    El input empieza con una línea que contiene un entero $T$, donde $1\leq T \leq
    100$, denotando el número de casos de prueba. Luego $T$ líneas siguen, cada
    una conteniendo un caso de prueba. Cada caso de prueba consiste de dos enteros $A,B$,
    donde $-2^{20} \leq A,B \leq 2^{20}$, separados por un espacio.
    \end{block}

    \vspace{10pt}

    \begin{block}{Output}
    Por cada caso de prueba, imprima una línea conteniendo el valor de $A\times B$.
    \end{block}
\end{frame}

\begin{frame}{Ejemplo de un problema}
    \begin{center}
        \begin{tabular}{|l|l|}
            \hline
            {\footnotesize Input de ejemplo} & {\footnotesize Output de ejemplo} \\
            \hline
            \begin{minipage}{80pt}
\vspace{10pt}
\ttfamily
4\\
3 4\\
13 0\\
1 8\\
100 100\\
            \end{minipage}
&
\begin{minipage}{80pt}
\vspace{10pt}
\ttfamily
12\\
0\\
8\\
10000\\
\end{minipage}
\\
            \hline
        \end{tabular}
    \end{center}

\end{frame}

\begin{frame}[fragile]{Ejemplo de solución}
    \begin{minted}[fontsize=\scriptsize]{cpp}
#include <iostream>
using namespace std;

int main() {
    int T;
    cin >> T;

    for (int t = 0; t < T; t++) {

        int A, B;
        cin >> A >> B;

        cout << A * B << endl;
    }

    return 0;
}
\end{minted}

    \bi
        \onslide<2->{\item ¿La solución es correcta? \onslide<5->{\alert{¡No!}}}
        \onslide<3->{\item ¿Y si $A = B = 2^{20}$? \onslide<4->{El output es $0$...}}
    \ei
\end{frame}

\begin{frame}[fragile]{Ejemplo de solución}
    \bi
        \item Cuando $A = B = 2^{20}$, la respuesta debería ser $2^{40}$
        \onslide<2->{\item Muy grande para un entero de 32 bits, overflow}
        \onslide<3->{\item Un entero de 64 debería ser suficiente}
    \ei
\end{frame}

\begin{frame}[fragile]{Ejemplo de solución}
    \begin{minted}[fontsize=\scriptsize]{cpp}
#include <iostream>
using namespace std;

int main() {
    int T;
    cin >> T;

    for (int t = 0; t < T; t++) {

        long long A, B;
        cin >> A >> B;

        cout << A * B << endl;
    }

    return 0;
}
\end{minted}

    \bi
    \onslide<2->{\item ¿La solución es correcta? \onslide<3->{{\alert{¡Si!}}}}
    \ei
\end{frame}

\begin{frame}[fragile]{Errores comunes}
    \begin{minted}[fontsize=\scriptsize]{cpp}
#include <iostream>
using namespace std;

int main() {
    int T;
    cout << "Ingrese el numero de casos de prueba: ";
    cin >> T;

    for (int t = 0; t < T; t++) {

        long long A, B;
        cout << "Ingrese dos enteros: ";
        cin >> A >> B;

        // Verificar limites -2^20 <= A, B <= 2^20
        if (-(1<<20) <= A && A <= (1<<20) &&
            -(1<<20) <= B && B <= (1<<20)) {

            cout << "La respuesta es: "; 
            cout << A * B << endl;
        }
    }

    return 0;
}
\end{minted}

\end{frame}

\begin{frame}{Jueces virtuales}
    \bi
        \item De los jueces más populares:
        \bi
            \item \alert{Codeforces}
            \item \alert{AtCoder}
            \item \alert{Kattis}
            \item \alert{Online Judge (ex UVa)}
        \ei
        \item Se envían las soluciones a los jueces virtuales y se obtiene feedback de inmediato
        \item Se puede enviar en cualquier lenguaje soportado:
            \bi
                \item C
                \item C++
                \item Java
                \item Python 2
                \item Python 3
                \item C\#{}
                \item y otros
            \ei
    \ei
\end{frame}

\begin{frame}{Veredictos}
    \bi
        \item El feedback sobre las soluciones es limitado
        \item Usualmente se recibe alguno de los siguientes veredictos:
            \bi
                \item Accepted
                \item Wrong Answer
                \item Compile Error
                \item Run Time Error
                \item Time Limit Exceeded
                \item Memory Limit Exceeded
            \ei

        \item No se revela cuáles son los casos de prueba que se usan para probar la solución
    \ei
\end{frame}

\section{Tips para competencias}

\begin{frame}{Escribir más rápido}
    \bi
        \item Buenos competidores tienen soluciones simples; no tienen que escribir tanto, pero es importante escribir rápido
        \item A veces es inevitable tener que escribir mucho código
        \vspace{15pt}
        \item Typing.com tiene un curso de mecanografía gratuito:
        \item \textbf{https://www.typing.com/student/lessons}
        \vspace{15pt}
        \item TypeRacer es una forma divertida y efectiva de practicar:
        \item \textbf{https://play.typeracer.com/}
        \vspace{15pt}
        \item Monketype es una alternativa con más opciones de configuración:
        \item \textbf{https://monkeype.com/}
    \ei
\end{frame}

\begin{frame}{Análisis de algoritmos}
    \bi
        \item Al resolver un problema, nuestra solución debe ser lo suficientemente rápida y no puede usar demasiada memoria
        \item También queremos que nuestra solución sea lo más simple posible
        \vspace{5pt}
        \item Debemos analizar el algoritmo para determinar si una solución se ejecutará dentro del tiempo límite
        \item Regla general: $10^{8}$ operaciones por segundo
        \vspace{10pt}
        \item<2-> Queremos ordenar $n \leq 10^{6}$ enteros, y tenemos 3 segundos.
            \bi
                \item ¿Podemos usar un simple bubble sort? $O(n^2)$
                \item ¿Qué hay de un merge sort? $O(n\log n)$
            \ei
        \vspace{5pt}
        \item<3-> Queremos ordenar $n \leq 10^{3}$ enteros, y tenemos 3 segundos.
            \bi
                \item ¿Podemos usar un simple bubble sort? $O(n^2)$
            \ei
        \vspace{5pt}
        \item<4-> Siempre ir por la solución más simple que no sobrepase el tiempo límite
    \ei
\end{frame}

\begin{frame}{Análisis de algoritmos}
    \bi
        \item A veces no estás seguro si tu solución es correcta
        \item ¡Intenta \textit{demostrar} su correctitud!
        \item Incluso si no logras demostrarlo o desmentirlo, probablemente tendrás una mejor comprensión del problema
        \vspace{20pt}
        \item Libro gratuito sobre demostraciones: \alert{\href{https://www.people.vcu.edu/~rhammack/BookOfProof/}{Book of Proof}}
    \ei
\end{frame}

\begin{frame}{Aprende tu lenguaje de programación}
    \bi
        \item Lo \textit{ideal} es conocer el lenguaje de programación como la palma de tu mano
        \item Esto incluye sus librerías:
            \bi
                \item C++ --- Standard Template Library
                \item Java --- Class Library
            \ei
        \item Si el lenguaje ya lo tiene implementado, no es necesario hacerlo de nuevo
    \ei
\end{frame}

\begin{frame}{Prueba tu solución}
    \bi
        \item Verificar que la solución es correcta y que corre dentro del tiempo límite 
        \item O ver que no es correcta aunque no sepas por qué
        \vspace{10pt}
        \item \textbf{Trata de romper tu solución} encontrando un contraejemplo 
        (una entrada para la cual tu solución da una salida incorrecta, 
        o tarda demasiado en calcular una respuesta)
        \item Trata con casos borde, input grande, etc.
    \ei
\end{frame}

\begin{frame}{Práctica y más práctica}
    \bi
        \item La habilidad de resolver problemas viene con la práctica
        \item Muchas veces se resuelven problemas encontrando patrones (el problema es similar a otro que ya resolviste)
        \item Muchos jueces online tienen problemas de competencias pasadas
        \item Algunos de estos jueces organizan competencias frecuentemente
        \item Codeforces, AtCoder, Codechef, LeetCode, TopCoder, etc.
    \ei
\end{frame}

\begin{frame}{Formato de competencias}
    \bi
        \item Contests duran 2 días 
        \item Participación individual o en equipo
        \item Necesitan una cuenta en \alert{\href{https://vjudge.net}{vjudge.net}}
    \ei
\end{frame}

\end{document}
