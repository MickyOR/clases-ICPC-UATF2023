\documentclass[10pt]{beamer}
\usepackage[utf8]{inputenc}
\usepackage[T1]{fontenc}
\usetheme{metropolis}
\usepackage{booktabs}
\usepackage[scale=2]{ccicons}
\usepackage{pgfplots}
\usepgfplotslibrary{dateplot}
\usepackage{xspace}
\usepackage{pbox}

% graphics path
\graphicspath{{./images/}}

% a few macros
\newcommand{\bi}{\begin{itemize}}
\newcommand{\ei}{\end{itemize}}
\newcommand{\ig}{\includegraphics}

% title info
\title{Estructuras de datos no lineales implementadas en librerías}
\author{Miguel Ortiz}
\institute{Octubre 2023}
\date{\textbf{Programación competitiva para ICPC}}

% Tikz
\usepackage{tikz}
\usetikzlibrary{arrows,shapes}

% Minted
\usepackage{minted}
\usemintedstyle{manni}
\newminted{cpp}{fontsize=\footnotesize}

% Graph styles
\tikzstyle{vertex}=[circle,fill=black!50,minimum size=15pt,inner sep=0pt, font=\small]
\tikzstyle{selected vertex} = [vertex, fill=red!24]
\tikzstyle{edge} = [draw,thick,-]
\tikzstyle{dedge} = [draw,thick,->]
\tikzstyle{weight} = [font=\scriptsize,pos=0.5]
\tikzstyle{selected edge} = [draw,line width=2pt,-,red!50]
\tikzstyle{ignored edge} = [draw,line width=5pt,-,black!20]


\begin{document}
\maketitle

\begin{frame}{Introducción}
    \bi
        \item Una estructura de datos es una forma de almacenar datos en una computadora
        \item Es importante elegir la estructura de datos apropiada para cada problema
        \item Estructuras de datos sofisticadas y poderosas ya están implementadas en los lenguajes más populares
        \item Veremos algunas de las más importantes presentes en C++
    \ei
\end{frame}

\begin{frame}[fragile]{Arreglos dinámicos - lineal}
  \bi
    \item Un arreglo dinámico puede cambiar su tamaño durante la ejecución del programa
    \item \mintinline{cpp}|vector| en C++
    \item Puede usarse casi como un arreglo normal
  \ei
\end{frame}

\begin{frame}[fragile]{Arreglos dinámicos - lineal}
  Crear un vector vacío y añadir elementos:
  \begin{minted}[]{cpp}
    #include <vector>
    ...
    vector<int> v;
    v.push_back(3); // [3]
    v.push_back(2); // [3, 2]
    v.push_back(5); // [3, 2, 5]
  \end{minted}
  \onslide<2->{
    Se puede acceder a los elementos como en un arreglo normal:
  }
  \onslide<2->\begin{minted}[]{cpp}
    cout << v[0] << '\n'; // 3
    cout << v[1] << '\n'; // 2
    cout << v[2] << '\n'; // 5
  \end{minted}
\end{frame}

\begin{frame}[fragile]{Arreglos dinámicos - lineal}
  La función \mintinline{cpp}|size| devuelve el tamaño del vector:
  \begin{minted}[]{cpp}
    for (int i = 0; i < v.size(); i++) {
      cout << v[i] << " ";
    }
    // 3 2 5
  \end{minted}
  Una forma más corta de iterar sobre los elementos:
  \begin{minted}[]{cpp}
    for (auto x : v) {
      cout << x << " ";
    }
    // 3 2 5
  \end{minted}
\end{frame}

\begin{frame}[fragile]{Arreglos dinámicos - lineal}
  Inicializar un vector con 5 elementos:
  \begin{minted}[]{cpp}
    vector<int> v = {3,2,5,1,4};
  \end{minted}
  También se puede inicializar con un tamaño y un valor en cada posición:
  \begin{minted}[]{cpp}
    // tamaño 10, lleno de 0s
    vector<int> v(10); 
  \end{minted}
  \begin{minted}[]{cpp}
    // tamaño 10, lleno de -1s
    vector<int> v(10, -1); 
  \end{minted}
  \begin{minted}[]{cpp}
    // tamaño 5, lleno de vectores vacios
    vector<vector<int>> v(5, vector<int>()); 
  \end{minted}
\end{frame}

\begin{frame}[fragile]{Arreglos dinámicos - lineal}
  Otras funciones:
  \bi
    \item \mintinline{cpp}|v.pop_back()|: elimina el último elemento
    \item \mintinline{cpp}|v.resize(n)|: cambia el tamaño del vector a $n$
    \item \mintinline{cpp}|v.assign(n, val)|: cambia el tamaño del vector a $n$ y llena todas las posiciones con $val$
    \item \mintinline{cpp}|v.clear()|: elimina todos los elementos del vector
  \ei
  Ordenar un vector:
  \begin{minted}[]{cpp}
    #include <algorithm>
    ...
    sort(v.begin(), v.end());
  \end{minted}
\end{frame}

\begin{frame}{Binary Heap - Introducción}
  \bi
    \item Organiza los datos en un árbol binario \textit{completo}
    \item Por cada nodo intermedio, el valor del nodo es mayor o igual que el valor de sus hijos
    \item El nodo raíz tiene el mayor valor de todos
    \item Los valores $\{1, 2, 3, 7, 17, 19, 25, 36, 90\}$ pueden generar el siguiente binary heap:
  \ei
  \begin{center}
    \ig[width=\textwidth]{BinaryHeap.png}
  \end{center}
\end{frame}

\begin{frame}{Binary Heap - Cola de prioridad}
  \bi
    \item Operaciones 
    \bi
      \item Insertar un elemento (\mintinline{cpp}|push|) en $O(\log n)$
      \item Obtener el elemento con mayor valor (\mintinline{cpp}|top|) en $O(1)$
      \item Eliminar el elemento con mayor valor (\mintinline{cpp}|pop|) en $O(\log n)$
    \ei
    \vspace{10pt}
  \ei
  \bi
    \item Nosotros definimos el tipo de dato que almacena el binary heap
    \item \mintinline{cpp}|priority_queue<int> pq;|
    \item \mintinline{cpp}|priority_queue<pair<int, int>> pq;|
  \ei
\end{frame}

\begin{frame}[fragile]{Binary Heap - Cola de prioridad}
  \bi
    \item \mintinline{cpp}|push|, \mintinline{cpp}|top| y \mintinline{cpp}|pop|:
  \ei
  \begin{minted}[]{cpp}
    #include <queue>
    ...
    priority_queue<int> pq;
    pq.push(3);
    pq.push(2);
    pq.push(5);
    pq.push(3);
    cout << pq.top() << '\n'; // 5
    pq.pop();
    cout << pq.top() << '\n'; // 3
  \end{minted}
\end{frame}

\begin{frame}[fragile]{Binary Heap - Cola de prioridad}
  \bi
    \item Manatener un binary heap con el menor elemento en el tope:
  \ei
  \begin{minted}[]{cpp}
    ...
    priority_queue<int, vector<int>, greater<int>> pq;
    pq.push(3);
    pq.push(2);
    pq.push(5);
    pq.push(3);
    cout << pq.top() << '\n'; // 2
    pq.pop();
    cout << pq.top() << '\n'; // 3
  \end{minted}
\end{frame}

\begin{frame}{Hash Table}
  \bi
    \item Estructura de datos que almacena pares de elementos (llave, valor)
    \item Usamos la llave para acceder al valor
    \item Se usa \mintinline{cpp}|unordered_map<tipo_llave, tipo_valor>| 
    cuando no importa el orden de los elementos
    \item Se usa \mintinline{cpp}|unordered_set<tipo_llave>| para verificar 
    la existencia de llaves cuando no importa el orden de los elementos
    \onslide<2>\item Insertar, buscar/acceder y eliminar un elemento es $O(1)$ usando una función hash
    \ei
  \onslide<2>\begin{center}
    \ig[width=0.5\textwidth]{HashTable.png}
  \end{center}
\end{frame}

\begin{frame}[fragile]{Hash Table - \mintinline{cpp}|unordered_map|}
  \bi
    \item \mintinline{cpp}|insert|, \mintinline{cpp}|count|, \mintinline{cpp}|erase| 
    y \mintinline{cpp}|clear| en \mintinline{cpp}|unordered_map|:
  \ei
  \begin{minted}[]{cpp}
    #include <unordered_map>
    ...
    unordered_map<string, int> m;
    m["mono"] = 3;
    m["gato"] = 5;
    cout << m["mono"] << '\n'; // 3
    if (!m.count("tecla")) {
      // llave no existe
    }
    cout << m["tecla"] << '\n'; // 0
    if (m.count("tecla")) {
      // llave existe
    }
    m.erase("mono");
    m.clear();
  \end{minted}
\end{frame}

\begin{frame}[fragile]{Hash Table - \mintinline{cpp}|unordered_set|}
  \bi
    \item \mintinline{cpp}|insert|, \mintinline{cpp}|count|, \mintinline{cpp}|erase| 
    y \mintinline{cpp}|clear| en \mintinline{cpp}|unordered_set|:
  \ei
  \begin{minted}[]{cpp}
    #include <unordered_set>
    ...
    unordered_set<string> s;
    s.insert("mono");
    s.insert("gato");
    cout << s.count("mono") << '\n'; // 1
    cout << s.count("tecla") << '\n'; // 0
    s.erase("mono");
    cout << s.size() << '\n'; // 1
    s.clear();
  \end{minted}
\end{frame}

\begin{frame}[fragile]{Hash Table}
  \bi
    \item Recorrer un \mintinline{cpp}|unordered_map| y un \mintinline{cpp}|unordered_set|:
  \ei
  \begin{minted}[]{cpp}
    unordered_map<string, int> m;
    m["mono"] = 3; m["gato"] = 5;
    for (auto p : m) { 
      cout << p.first << " " << p.second << '\n'; 
    }
    // mono 3
    // gato 5

    unordered_set<string> s;
    s.insert("arbol"); s.insert("planta");
    for (auto x : s) { cout << x << '\n'; }
    // planta
    // arbol
  \end{minted}
\end{frame}

\begin{frame}{Binary Search Tree}
  \bi
    \item Similarmente al binary heap, organiza los datos en un árbol binario
    \item Por cada nodo intermedio, el valor del nodo es mayor que el valor de su hijo izquierdo y menor que el valor de su hijo derecho
  \ei
  \begin{center}
    \ig[width=\textwidth]{BST.png}
  \end{center}
  \onslide<2>\bi
    \item Permite insertar, buscar y eliminar un elemento en $O(\log n)$
    si el árbol está \textit{balanceado}
    \item C++ tiene dos implementaciones: \mintinline{cpp}|set| y \mintinline{cpp}|map|
  \ei
\end{frame}

\begin{frame}{Binary Search Tree - \mintinline{cpp}|set|}
  \bi
    \item Un \mintinline{cpp}|set| almacena un conjunto de elementos (sin repeticiones)
    \item Se puede insertar, eliminar y verificar si un elemento está en el conjunto
    \item Al ser un árbol binario de búsqueda, mantiene los elementos ordenados
  \ei
\end{frame}

\begin{frame}[fragile]{Binary Search Tree - \mintinline{cpp}|set|}
  \bi
    \item \mintinline{cpp}|insert|, \mintinline{cpp}|count| y \mintinline{cpp}|erase|:
  \ei
  \begin{minted}[]{cpp}
    #include <set>
    ...
    set<int> s;
    s.insert(3);
    s.insert(2);
    s.insert(5);
    s.insert(3); // no se inserta
    cout << s.count(3) << '\n'; // 1
    cout << s.count(4) << '\n'; // 0
    cout << s.size() << '\n'; // 3
    s.erase(3);
    cout << s.count(3) << '\n'; // 0
    cout << s.size() << '\n'; // 2
  \end{minted}
\end{frame}

\begin{frame}[fragile]{Binary Search Tree - \mintinline{cpp}|set|}
  \bi
    \item Recorrer un \mintinline{cpp}|set|:
  \ei
  \begin{minted}[]{cpp}
    set<int> s = {2, 4, 1, 6, 1};
    for (auto x : s) {
      cout << x << " ";
    }
    // 1 2 4 6
  \end{minted}
\end{frame}

\begin{frame}[fragile]{Binary Search Tree - \mintinline{cpp}|set|}
  \bi
    \item \mintinline{cpp}|set| tambien permite hallar el elemento más cercano a un valor dado
    \item \mintinline{cpp}|lower_bound| devuelve un iterador (puntero) al primer elemento \underline{mayor o igual} que el valor dado
    \item \mintinline{cpp}|upper_bound| devuelve un iterador (puntero) al primer elemento \underline{mayor} que el valor dado
  \ei
  \begin{minted}[]{cpp}
    set<int> s = {2, 4, 1, 6, 1};

    auto it_1 = s.lower_bound(3);
    cout << *it_1 << '\n'; // 4

    auto it_2 = s.upper_bound(4);
    cout << *it_2 << '\n'; // 6
  \end{minted}
\end{frame}

\begin{frame}[fragile]{Binary Search Tree - \mintinline{cpp}|set|}
  \bi
    \item Se puede mover un iterador a la izquierda o derecha usando los operadores \mintinline{cpp}|--| y \mintinline{cpp}|++|
  \ei
  \begin{minted}[]{cpp}
    set<int> s = {2, 4, 1, 6, 1};
    auto it = s.upper_bound(3); // apunta a 4
    it++; // apunta a 6
    cout << *it << '\n'; // 6
    it--; // apunta a 4
    it--; // apunta a 2
    cout << *it << '\n'; // 2
  \end{minted}
  % \begin{center}
  %   \begin{overprint}
  %     \begin{tabular}{|c|c|c|c|c|}
  %       1 & 2 & 4 & 6 & \\
  %       \hline
  %       s.begin() & & & & s.end() \\
  %     \end{tabular}
  %   \end{overprint}
  % \end{center}
  \onslide<2>\bi
    \item Podemos usar esto para hallar el elemento más cercano que sea menor o menor-igual que un valor dado
  \ei
\end{frame}

\begin{frame}[fragile]{Binary Search Tree - \mintinline{cpp}|set|}
  \begin{minted}[]{cpp}
    set<int> s = {2, 4, 1, 6, 1};

    // Elemento menor o igual a 4
    auto it_1 = s.upper_bound(4); 
    it--;
    cout << *it_1 << '\n'; // 4

    // Elemento menor a 4
    auto it_2 = s.lower_bound(4); 
    it--;
    cout << *it_2 << '\n'; // 2
  \end{minted}
\end{frame}

\begin{frame}[fragile]{Conjuntos con elementos repetidos}
  \bi
    \item C++ también contiene \mintinline{cpp}|multiset| y \mintinline{cpp}|unordered_multiset|
    \item La única diferencia es que permiten elementos repetidos
  \ei
  \begin{minted}[]{cpp}
    multiset<int> s;
    s.insert(5);
    s.insert(5);
    s.insert(5);
    cout << s.count(5) << '\n'; // 3
  \end{minted}
  \bi
    \item Se debe tener cuidado si solo se quiere eliminar un elemento
  \ei
  \begin{minted}[]{cpp}
    s.erase(s.find(5));
    cout << s.count(5) << '\n'; // 2
    s.erase(5);
    cout << s.count(5) << '\n'; // 0
  \end{minted}
\end{frame}

\begin{frame}[fragile]{Binary Search Tree - \mintinline{cpp}|map|}
  \bi
    \item Un \mintinline{cpp}|map| almacena pares de elementos (llave, valor)
    \item La llave puede ser de cualquier tipo que tenga definida la función \mintinline{cpp}|<| (que se pueda ordenar)
    \item \mintinline{cpp}|map| mantiene los elementos ordenados por su llave
  \ei
\end{frame}

\begin{frame}[fragile]{Binary Search Tree - \mintinline{cpp}|map|}
  \bi
    \item Insertar y acceder a elementos:
  \ei
  \begin{minted}[]{cpp}
    #include <map>
    ...
    map<string, int> m;
    m["mono"] = 3;
    m["gato"] = 5;
    m["cuaderno"] = 2;
    cout << m["mono"] << '\n'; // 3
    cout << m["tecla"] << '\n'; // 0
  \end{minted}
\end{frame}

\begin{frame}[fragile]{Binary Search Tree - \mintinline{cpp}|map|}
  \bi
    \item \mintinline{cpp}|count|, \mintinline{cpp}|erase| y recorrer un \mintinline{cpp}|map|:
  \ei
  \begin{minted}[]{cpp}
    if (m.count("tecla")) {
      // llave existe
    }
    m.erase("mono");
    for (auto p : m) {
      // p.first  -> llave
      // p.second -> valor
      cout << p.first << " " << p.second << '\n';
    }
    // cuaderno 2
    // gato 5
    // tecla 0
  \end{minted}
\end{frame}

\begin{frame}[fragile]{Consideraciones}
  \bi
    \item Las operaciones de \mintinline{cpp}|unordered_map| y \mintinline{cpp}|unordered_set| 
    son, generalmente, $O(1)$, a diferencia de \mintinline{cpp}|map| y \mintinline{cpp}|set| 
    donde son $O(\log n)$. Siempre usar las versiones que usen un hash table a menos que 
    sea necesario ordenar las llaves.
    \item Si un tipo de dato no tiene implementada una función hash 
    (como \mintinline{cpp}|pair<int, int>|), no se puede usar como llave
    en un \mintinline{cpp}|unordered_map|.
  \ei
\end{frame}

\end{document}
