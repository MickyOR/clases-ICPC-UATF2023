\documentclass[10pt]{beamer}
\usepackage[utf8]{inputenc}
\usepackage[T1]{fontenc}
\usetheme{metropolis}
\usepackage{booktabs}
\usepackage[scale=2]{ccicons}
\usepackage{pgfplots}
\usepgfplotslibrary{dateplot}
\usepackage{xspace}
\usepackage{pbox}

% graphics path
\graphicspath{{./images/}}

% a few macros
\newcommand{\bi}{\begin{itemize}}
\newcommand{\ei}{\end{itemize}}
\newcommand{\ig}{\includegraphics}

% title info
\title{Estructuras de datos}
\author{Miguel Ortiz}
\institute{Abril 2023 - Cochabamba, Bolivia}
\date{\textbf{Programación competitiva para ICPC}}

% Tikz
\usepackage{tikz}
\usetikzlibrary{arrows,shapes}

% Minted
\usepackage{minted}
\usemintedstyle{manni}
\newminted{cpp}{fontsize=\footnotesize}

% Graph styles
\tikzstyle{vertex}=[circle,fill=black!50,minimum size=15pt,inner sep=0pt, font=\small]
\tikzstyle{selected vertex} = [vertex, fill=red!24]
\tikzstyle{edge} = [draw,thick,-]
\tikzstyle{dedge} = [draw,thick,->]
\tikzstyle{weight} = [font=\scriptsize,pos=0.5]
\tikzstyle{selected edge} = [draw,line width=2pt,-,red!50]
\tikzstyle{ignored edge} = [draw,line width=5pt,-,black!20]


\begin{document}
\maketitle

\begin{frame}{Introducción}
    \bi
        \item Una estructura de datos es una forma de almacenar datos en una computadora
        \item Es importante elegir la estructura de datos apropiada para cada problema
        \item Estructuras de datos sofisticadas y poderosas ya están implementadas en los lenguajes más populares
        \item Veremos algunas de las más importantes presentes en C++
    \ei
\end{frame}

\begin{frame}[fragile]{Arreglos dinámicos}
  \bi
    \item Un arreglo dinámico puede cambiar su tamaño durante la ejecución del programa
    \item \mintinline{cpp}|vector| en C++
    \item Puede usarse casi como un arreglo normal
  \ei
\end{frame}

\begin{frame}[fragile]{Arreglos dinámicos}
  Crear un vector vacío y añadir elementos:
  \begin{minted}[]{cpp}
    #include <vector>
    ...
    vector<int> v;
    v.push_back(3); // [3]
    v.push_back(2); // [3, 2]
    v.push_back(5); // [3, 2, 5]
  \end{minted}
  \onslide<2->{
    Se puede acceder a los elementos como en un arreglo normal:
  }
  \onslide<2->\begin{minted}[]{cpp}
    cout << v[0] << '\n'; // 3
    cout << v[1] << '\n'; // 2
    cout << v[2] << '\n'; // 5
  \end{minted}
\end{frame}

\begin{frame}[fragile]{Arreglos dinámicos}
  La función \mintinline{cpp}|size| devuelve el tamaño del vector:
  \begin{minted}[]{cpp}
    for (int i = 0; i < v.size(); i++) {
      cout << v[i] << " ";
    }
    // 3 2 5
  \end{minted}
  Una forma más corta de iterar sobre los elementos:
  \begin{minted}[]{cpp}
    for (auto x : v) {
      cout << x << " ";
    }
    // 3 2 5
  \end{minted}
\end{frame}

\begin{frame}[fragile]{Arreglos dinámicos}
  Inicializar un vector con 5 elementos:
  \begin{minted}[]{cpp}
    vector<int> v = {3,2,5,1,4};
  \end{minted}
  También se puede inicializar con un tamaño y un valor en cada posición:
  \begin{minted}[]{cpp}
    // tamaño 10, lleno de 0s
    vector<int> v(10); 
  \end{minted}
  \begin{minted}[]{cpp}
    // tamaño 10, lleno de -1s
    vector<int> v(10, -1); 
  \end{minted}
  \begin{minted}[]{cpp}
    // tamaño 5, lleno de vectores vacios
    vector<vector<int>> v(5, vector<int>()); 
  \end{minted}
\end{frame}

\begin{frame}[fragile]{Arreglos dinámicos}
  Otras funciones:
  \bi
    \item \mintinline{cpp}|v.pop_back()|: elimina el último elemento
    \item \mintinline{cpp}|v.resize(n)|: cambia el tamaño del vector a $n$
    \item \mintinline{cpp}|v.assign(n, val)|: cambia el tamaño del vector a $n$ y llena todas las posiciones con $val$
    \item \mintinline{cpp}|v.clear()|: elimina todos los elementos del vector
  \ei
  Ordenar un vector:
  \begin{minted}[]{cpp}
    #include <algorithm>
    ...
    sort(v.begin(), v.end());
  \end{minted}
\end{frame}

\begin{frame}[fragile]{Arreglos dinámicos}
  \bi
  \item Un \mintinline{cpp}|string| también es un arreglo dinámico
  \item Almacena caracteres y tiene funciones para manipularlos
  \ei
  \begin{minted}[]{cpp}
    string a = "abc";
    string b = a + a;
    cout << b << '\n'; // abcabc
    b[2] = 'x';
    cout << b << '\n'; // abxabc
    string c = b.substr(2, 3);
    cout << c << '\n'; // xab
    c.pop_back();
    cout << c << '\n'; // xa
  \end{minted}
\end{frame}

\begin{frame}[fragile]{Conjuntos}
  \bi
    \item Un \mintinline{cpp}|set| es una estructura de datos que almacena un conjunto de elementos (sin repeticiones)
    \item Se puede insertar y eliminar elementos, y verificar si un elemento está en el conjunto
    \item C++ tiene dos tipos de sets: \mintinline{cpp}|set| y \mintinline{cpp}|unordered_set|
    \item \mintinline{cpp}|set| está implementado como un árbol binario de búsqueda balanceado (operaciones en $O(\log n)$)
    \item \mintinline{cpp}|set| mantiene los elementos ordenados
    \item \mintinline{cpp}|unordered_set| está implementado como una tabla de hash (operaciones en $O(1)$ en promedio)
  \ei
\end{frame}

\begin{frame}[fragile]{Conjuntos}
  \bi
    \item La función \mintinline{cpp}|insert| añade un elemento al conjunto
    \item La función \mintinline{cpp}|count| devuelve el número de ocurrencias de un elemento en el conjunto (0 o 1)
    \item La función \mintinline{cpp}|erase| elimina un elemento del conjunto
  \ei
  \begin{minted}[]{cpp}
    #include <set>
    ...
    set<int> s;
    s.insert(3);
    s.insert(2);
    s.insert(5);
    s.insert(3); // no se inserta
    cout << s.count(3) << '\n'; // 1
    cout << s.count(4) << '\n'; // 0
    s.erase(3);
    cout << s.count(3) << '\n'; // 0
  \end{minted}
\end{frame}

\begin{frame}[fragile]{Conjuntos}
  \bi
    \item A diferencia de los vectores, no se puede acceder a los elementos de un conjunto por su índice
    \item Se puede iterar sobre los elementos de un conjunto de forma similar
    \item Función \mintinline{cpp}|size| devuelve el número de elementos en el conjunto
  \ei
  \begin{minted}[]{cpp}
    set<int> s = {2, 5, 6, 8};
    cout << s.size() << '\n'; // 4
    for (auto x : s) {
      cout << x << " ";
    }
  \end{minted}
\end{frame}

\begin{frame}[fragile]{Conjuntos}
  \bi
    \item C++ también contiene \mintinline{cpp}|multiset| y \mintinline{cpp}|unordered_multiset|
    \item La única diferencia es que permiten elementos repetidos
  \ei
  \begin{minted}[]{cpp}
    multiset<int> s;
    s.insert(5);
    s.insert(5);
    s.insert(5);
    cout << s.count(5) << '\n'; // 3
  \end{minted}
  \bi
    \item Se debe tener cuidado si solo se quiere eliminar un elemento
  \ei
  \begin{minted}[]{cpp}
    s.erase(s.find(5));
    cout << s.count(5) << '\n'; // 2
    s.erase(5);
    cout << s.count(5) << '\n'; // 0
  \end{minted}
\end{frame}

\begin{frame}[fragile]{Mapas}
  \bi
    \item Un \mintinline{cpp}|map| es una estructura de datos que almacena pares de elementos (llave, valor)
    \item En un arreglo se accede a los elementos por su índice, en un mapa se accede por su llave
    \item La llave puede ser de cualquier tipo
    \item Hay dos implementaciones de mapas: \mintinline{cpp}|map| y \mintinline{cpp}|unordered_map|
    \item Un \mintinline{cpp}|map| está implementado como un árbol binario de búsqueda balanceado (operaciones en $O(\log n)$)
    \item \mintinline{cpp}|map| mantiene los elementos ordenados por su llave
    \item \mintinline{cpp}|unordered_map| está implementado como una tabla de hash (operaciones en $O(1)$ en promedio)
  \ei
\end{frame}

\begin{frame}[fragile]{Mapas}
  \bi
    \item Se añaden elementos usando la llave como si ya existiera en el mapa
    \item Si se intenta acceder a un elemento que no existe, se crea un nuevo elemento con valor por defecto (0 para enteros)
  \ei
  \begin{minted}[]{cpp}
    #include <map>
    ...
    map<string, int> m;
    m["mono"] = 3;
    m["gato"] = 5;
    m["cuaderno"] = 2;
    cout << m["mono"] << '\n'; // 3
    cout << m["tecla"] << '\n'; // 0
  \end{minted}
\end{frame}

\begin{frame}[fragile]{Mapas}
  \bi
    \item La función \mintinline{cpp}|count| devuelve el número de ocurrencias de una llave en el mapa (0 o 1)
    \item La función \mintinline{cpp}|erase| elimina un elemento del mapa
    \item Se puede iterar sobre los elementos de un mapa de forma similar a un \mintinline{cpp}|set|
  \ei
  \begin{minted}[]{cpp}
    if (m.count("tecla")) {
      // llave existe
    }
    m.erase("mono");
    for (auto p : m) {
      cout << p.first << " " << p.second << '\n';
    }
    // cuaderno 2
    // gato 5
    // tecla 0
  \end{minted}
\end{frame}

\end{document}
